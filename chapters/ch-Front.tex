\ProvidesFile{ch-Front.tex}[2022-10-05 front matter chapter]
%
%  This is the ``front matter'' for the thesis.
%
%  REFERENCES
%
%    TCMOS17
%      The Chicago Manual of Style Online, 17th edition.
%      https://www.chicagomanualofstyle.org/home.html
%      retrieved on 2020-02-29
%
%    TEMPL
%      Thesis and Disertation Office Templates.
%      https://www.purdue.edu/gradschool/research/thesis/templates.html
%      retrieved on 2020-02-29
%
%    WNNCD
%    Webster's Ninth New Collegiate Dictionary.
%

%
%   Only Purdue University uses this page
%
%   Comment out \begin{statement} through \end{statement}
%   if you are not at Purdue University.
%
% Statement of Thesis/Dissertation Approval Page
% This page is REQUIRED.  The page should be numbered "2"
% and should NOT be listed in your TABLE OF CONTENTS.
\begin{statement}
  % Delete or add \entry commands as needed for all committe members.
  \entry{Dr.~Andreas Jung, Chair}{Department of Physics and Astronomy}
  \entry{Dr.~John Finley}{Department of Physics and Astronomy}
  \entry{Dr.~Matthew Jones}{Department of Physics and Astronomy}
  \entry{Dr.~Martin Kruczenski}{Department of Physics and Astronomy}

  % There should be one \approvedby command containing the
  % "FORM 9 THESIS FORM HEAD NAME HERE" (from TEMPL, retrieved on 2020-03-01).
  \approvedby{Dr.~Gabor Csathy}{\centering Head of the Department of Physics and Astronomy\par}
\end{statement}

% Dedication page is optional.
% A name and often a message in tribute to a person or cause.
% References: WEB9 332.
%\begin{dedication}
%  To graduate students
%\end{dedication}

% Acknowledgements page is optional but most theses include
% a brief statement of appreciation or recognition of special
% assistance.
\begin{acknowledgments}
I express gratitude to my thesis advisor, Andreas Jung, for his guidance throughout my research. His expertise, encouragement, and feedback have been invaluable in shaping this work. 
I also thank Gabriele Benelli, Ron Lipton, and Corrinne Mills for the opportunities to collaborate under their supervision and learn from their expertise.

Acknowledgement is due to the other members of the top quark research group at Purdue, including Amandeep Bakshi, Giulia Negro, and Andrew Wildridge for their invaluable contributions to the measurements presented in this thesis. 
I would also like to recognize the top quark research group at DESY, Ajeeta Khatiwada, and Jacob Linacre for their instrumental role in developing the original measurement and unfolding framework that served as the foundation for this research.

My heartfelt appreciation is extended to the many colleagues and collaborators in the CMS experiment for their tireless efforts in maintaining and operating the CMS detector and for providing the high-quality data used for these measurements. 
Their exceptional expertise, hard work, and dedication have made this research possible.

I am also grateful to the members of my thesis committee, John Finley, Matthew Jones, and Martin Kruczenski, for their valuable time and constructive comments that have contributed to the quality and rigor of this dissertation.

Finally, I would like to thank my family and friends for their unwavering support, encouragement, and understanding throughout this challenging journey. 
Their love and care have been my source of strength and inspiration.
\end{acknowledgments}

% The preface is optional.
% References: TCMOS17 1.49, WEB9 927.
%\begin{preface}
%
%\end{preface}

% The Table of Contents is required.
% The Table of Contents will be automatically created for you
% using information you supply in
%     \chapter
%     \section
%     \subsection
%     \subsubsection
%     commands.
\pdfbookmark{TABLE OF CONTENTS}{Contents}
\tableofcontents

% If your thesis has tables, a list of tables is required.
% The List of Tables will be automatically created for you using
% information you supply in
%     \begin{table} ... \end{table}
% environments.
\listoftables

% If your thesis has figures, a list of figures is required.
% The List of Figures will be automatically created for you using
% information you supply in
%     \begin{figure} ... \end{figure}
% environments.
\listoffigures

% If your thesis has listings, a list of listings is required.
% The List of Listings will be automatically created for you using
% information you supply in
%     \begin{ZZlisting} ... \end{ZZlisting}
% environments.
%\ZZlistoflistings

% If your thesis has protocols, you may want to do a list of protocols.
% The List of Protocols will be automatically created for you using
% information you supply in
%     \begin{protocol} ... \end{protocol}
% environments.
%\listofprotocols

% If your thesis has schemes, you may want to do a list of schemes.
% The List of Schemes will be automatically created for you using
% information you supply in
%     \begin{scheme} ... \end{scheme}
% environments.
%\listofschemes

% List of Symbols is optional.
%\begin{symbols}
%  $+$& Positive Electric Charge\cr
%  $-$& Negative Electric Charge\cr  
%  $p$& Proton\cr
%  $n$& Neutron\cr
%  $\ell$& Lepton\cr
%  $e$& Electron\cr
%  $\mu$& Muon\cr
%  $\tau$ & Tau\cr
%  $\nu$& Neutrino\cr
%  $\gamma$& Photon\cr
%  $Z$& Z Boson\cr
%  $W^\pm$& W Boson\cr
%  $q$& Quark\cr
%  $u$& Up Quark\cr
%  $d$& Down Quark\cr
%  $c$& Charm Quark\cr
%  $s$& Strange Quark\cr
%  $t$& Top Quark\cr
%  $b$& Bottom Quark\cr
%  $g$& Gluon\cr
%  $R$& Red Color Charge\cr
%  $G$& Green Color Charge\cr
%  $B$& Blue Color Charge\cr
%  $H$& Higgs Boson\cr
%  $\bar{x}$& Anti-matter counterpart for arbitrary particle $x$\cr
%  $E$& Energy\cr
%  $m$& Mass\cr
%  $\pT$& Transverse Momentum\cr
%  $\eta$& Pseudorapidity\cr
%  $y$& Rapidity\cr
%  $\phi$& Azimuthal Angle\cr
%  $\MET$& Missing Transverse Energy\cr
%  $\Delta R$& Angular Separation in $\eta$-$\phi$ Space\cr
%  $\sqrt{s}$& Center-of-Mass Energy\cr
%  $\mathcal{L}$& Luminosity\cr
%  $\sigma$& Cross-section\cr
%  $N$& Number of Events\cr
%  $w$& Weight\cr
%  $SF$& Scale Factor\cr
%  & \cr
%  & \cr
%  & \cr
%  & \cr
%\end{symbols}

% List of Abbreviations is optional.
%\begin{abbreviations}
%  3DIC& 3-Dimensional Integrated Circuit\cr
%  BEH& Brout-Englert-Higgs\cr
%  BSM& Beyond the Standard Model\cr
%  CERN& European Center for Nuclear Research\cr
%  CHS& Charged-Hadron Subtraction\cr
%  CKM& Cabibbo-Kobayashi-Maskawa\cr
%  CMS& Compact Muon Solenoid\cr
%  CR& Color Reconnection\cr
%  DBI& Direct Bonding Interconnect\cr
%  DY& Drell-Yan\cr
%  EB& ECAL Barrel\cr
%  ECAL& Electromagnetic Calorimeter\cr
%  EDM& Event Data Model\cr
%  EE& ECAL End-cap\cr
%  EWSB& Electroweak Symmetry Breaking\cr
%  FEA& Finite Element Analysis\cr
%  FSR& Final State Radiation\cr
%  FNAL& Fermi National Accelerator Laborator\cr
%  GR& General Relativity\cr
%  GSF& Gaussian Sum Filter\cr
%  GUT& Grand Unification Theory\cr
%  GWS& Glashow-Weinberg-Salam\cr
%  HB& HCAL Barrel\cr
%  HCAL& Hadronic Calorimeter\cr
%  HE& HCAL End-cap\cr
%  HIPM& Heavily Ionizing Particle Mitigation\cr
%  IP& Interaction Point\cr
%  ISR& Initial State Radiation\cr
%  HL-LHC& High Luminosity Large Hadron Collider\cr
%  JEC& Jet Energy Correction\cr
%  JER& Jet Energy Resolution\cr
%  JES& Jet Energy Scale\cr
%  KF& Kalman Filter\cr
%  LGAD& Low Gain Avalanche Diode\cr
%  LHC& Large Hadron Collider\cr
%  LO& Leading-Order\cr
%  MC& Monte Carlo\cr
%  ME& Matrix Element\cr
%  MET& Missing Transverse Energy\cr
%  MIP& Minimum Ionizing Particle\cr
%  MPI& Multiple Parton Interactions\cr
%  MTD& MIP Timing Detector\cr
%  NLO& Next-to-Leading-Order\cr
%  NNLL& Next-to-Next-to-Leading-Logarithmic\cr
%  NNLO& Next-to-Next-to-Leading-Order\cr
%  NP& New Physics\cr
%  PDF& Parton Distribution Function\cr
%  PDG& Particle Data Group\cr  
%  PF& Particle-Flow\cr
%  PU& Pileup\cr
%  QED& Quantum Electrodynamics\cr  
%  QFT& Quantum Field Theory\cr
%  SF& Scale Factor\cr
%  SiPM& Silicon Photomultiplier\cr
%  SM& Standard Model of Particle Physics\cr
%  SMEFT& Standard Model Effective Field Theory\cr
%  SPICE& Simulation Program with Integrated Circuit Emphasis\cr
%  TCAD& Technology Computer-Aided Design\cr
%  TSV& Through-Silicon Via\cr
%  UE& Underlying Event\cr
%  WIMP& Weakly Interacting Massive Particle\cr
%  ZMF& Zero Momentum Frame\cr 
%  & \cr
%  & \cr
%  & \cr
%  & \cr
%  & \cr
%\end{abbreviations}


% Abstract is required.
% Note that the information for the first paragraph of the output
% doesn't need to be input here...it is put in automatically from
% information you supplied earlier using \title, \author, \degree,
% and \majorprof.
% Reference: PU 17.
\begin{abstract}%

This dissertation presents precision measurements of top quark ($t$ and $\bar{t}$) polarizations and top quark pair (\ttbar) spin correlations, which probe the independent coefficients of the top-spin components of the \ttbar production density matrix, targeting all channels (\ee, \emu, \mumu) of the \ttbar dileptonic decay mode with final states containing two oppositely charged leptons, and using \lumivalueRuniiUL $\pm$ \lumierrRuniiUL\ of data recorded by the CMS experiment at the LHC with \beamenergy, during 2016, 2017, and 2018.
All measured observables are corrected for detector efficiencies, acceptances, and migrations, unfolded to parton-level, and extrapolated to the full phase space using a regularized unfolding procedure.
Spin-density coefficients are extracted from the unfolded distributions and compared to theoretical predictions and predictions from Monte Carlo simulations with next-to-leading-order matrix element accuracy interfaced with parton-shower algorithms.
The measurements are performed both in the full phase-space and differentially as a function of \ttbar invariant mass.
The measured coefficients showed decent agreement with the MC predictions, and better agreement when compared to QCD perturbative calculations for \ttbar\ production at NLO with electroweak corrections, and the measurement precision for one-dimensional normalized unfolded cross-sections and extracted coefficients were improved by as much as a factor of two compared to previous measurements.

\end{abstract}

