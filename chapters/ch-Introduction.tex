\ProvidesFile{chapters/ch-Introduction.tex}

\chapter*{INTRODUCTION}
\addcontentsline{toc}{chapter}{INTRODUCTION}
\label{Introduction}
Experiments to discover and determine the properties of Standard Model (SM) particles are performed at particle accelerator facilities, such as the Large Hadron Collider (LHC) at CERN, which is the largest and most powerful particle accelerator ever built.
The measurements detailed in this dissertation are performed using \lumivalueRuniiUL $\pm$ \lumierrRuniiUL\ of data recorded by the Compact Muon Solenoid (CMS) experiment at the LHC with \beamenergy, during 2016, 2017, and 2018.

The LHC is considered a top quark factory capable of producing tens of millions of top quarks from proton-proton ($pp$) collisions every year.
Due to its unique properties and role in exotic processes, the top quark is important for precision measurements of the SM and is sensitive to new physics (NP) beyond the SM (BSM).
Top quark measurements are the only opportunity to study the properties of bare quarks, free of confinement effects.
With a very short life time caused by its exceptionally large mass, the top quark decays before hadronizing, and its properties, including spin information, are transferred to its decay products and observable in their kinematic distributions.

Measurements of \ttbar spin correlations in the dilepton mode have been performed by both CMS and ATLAS using $\SI{36}{\femto \b}$ of data taken during 2016 at \beamenergy~\cite{arxiv.1905.08634}\cite{Sirunyan:2681777}\cite{Aaboud:2667501}.
Both experiments unfolded their results to parton level and extrapolated to the full phase space.
CMS TOP-18-006, for which I collaborated as a coauthor, was published in Physics Review D and directly measured all the independent coefficients of the top-spin dependent parts of the \ttbar production density matrix in all dilepton channels.
Both CMS and ATLAS made indirect measurements of \ttbar spin correlations by measuring $\vert \Delta\phi_{\ell\bar{\ell}} \vert$, the difference in azimuthal angle between the two leptons in the laboratory frame, but ATLAS additionally measured $\vert \Delta\eta_{\ell\bar{\ell}} \vert$, the difference in pseudorapity between the two leptons in the laboratory frame, and performed their $\vert \Delta\phi_{\ell\bar{\ell}} \vert$ measurements differentially as a function of \ttbar invariant mass, but only made measurements in the \emu channel.
The ATLAS observation of a $3.2 \sigma$ deviation from the SM in their measurement of $\vert \Delta\phi_{\ell\bar{\ell}} \vert$ remains in tension with the CMS result, but could possibly be explained by missing higher-order corrections to the top quark kinematics in Monte Carlo (MC) simulations.

This dissertation presents differential measurements of top quark ($t$ and $\bar{t}$) polarizations and top quark pair (\ttbar) spin correlations, which probe the independent coefficients of the top-spin dependent parts of the \ttbar production density matrix.
The measurements target all channels (\ee, \emu, \mumu) of the \ttbar dileptonic decay mode with final states containing two oppositely charged leptons.
The measurements follow the analysis strategy of CMS TOP-18-006~\cite{Sirunyan:2681777}, but has been extended with additional observables, differential measurements as a function of \ttbar invariant mass, several optimizations that reduce systematic uncertainties, and significantly increased luminosity.
All measured distributions are corrected for detector efficiencies, acceptances, and migrations, unfolded to parton-level, and extrapolated to the full phase space using a regularized unfolding procedure with detector response obtained from MC simulated SM predictions with next-to-leading-order (NLO) matrix element accuracy with parton-shower algorithms.
The spin density coefficients are extracted from the unfolded distributions and compared to theoretical predictions, constituting a precision test of the SM.
The measurement results can be used for a multitude of subsequent analyses:
\begin{itemize}
    \item Effective field theory (EFT) interpretation to set limits on ten out of the eleven dimension-six operators relevant for hadronic \ttbar production~\cite{Sirunyan:2681777}
    \item Supersymmetry (SUSY) interpretation to set limits of production of light supersymmetric top squarks~\cite{CMS-PAS-FTR-18-034}
    \item Observe entanglement in the \ttbar system~\cite{Afik_2021}
    \item Observe a violation of Belle's Inequality~\cite{Aguilar_Saavedra_2022}
    \item Search for evidence of Toponium near \ttbar production threshold~\cite{PhysRevD.104.034023}
\end{itemize}

One of the ingredients for the \ttbar spin correlation measurements is a set of scale factors correcting the trigger efficiencies in MC simulation to those observed in data.
I performed a detailed measurement of these scale factors, which were approved by the EGamma and Muon CMS Physics Object Groups as well as by the CMS Top Quark Physics Analysis Group, and the results were centrally provided to be used by virtually all CMS Run II measurements analyzing \ttbar events decaying via the dileptonic mode.
In order to accommodate a search for Lorentz violation, I collaborated as a coauthor and additionally provided dedicated measurements of the dilepton trigger efficiency scale factors as a function of sidereal time.

My contributions to the operation of the CMS experiment to R\&D efforts for the upcoming high luminosity LHC (HL-LHC) phase are many-fold.
I have spent many weeks in the CMS Remote Operations Center at Fermi National Accelerator Laboratory (FNAL) monitoring offline the data quality of the CMS silicon pixel and strip sub-detectors and certifying collision data sets for tracking quality.
During the long shutdown phase in between Runs II and III, I was promoted to on-call expert, and later to shift leader, for the certification of ultra-legacy data sets.
Moreover, I developed data quality monitoring tools for the monitoring inefficient modules in the silicon strip sub-detector.

Also at FNAL, I contributed to the testing of HL-LHC pixel detector prototypes at the FNAL Test Beam Facility (FTBF).
The FTBF is an environment to probe the prototypes, consisting of RD-53a readout chips bump-bonded to planar or three-dimensional sensors, under controlled conditions.
Spills of several hundred thousand protons with energies up to $\SI{100}{\GeV}$ are provided by the accelerator facility every $\SI{60}{\s}$.
The CMS Tracking Telescope is placed along the beam line, and consists of silicon strip and pixel planes constructed from surplus silicon tracker modules of the same design as those currently installed in CMS.
The prototypes under testing are placed between two sections of planes in the telescope and their performance is probed with a set of beam tracks that are well-identified and reconstructed by the tracking telescope.
Besides data acquisition, I also contributed to the offline analysis of the data by identifying the reconstruction and alignment routines that were most effective, and automating these routines in a workflow with built-in quality checks that allowed us to efficiently align the telescope and produce measurement results for large quantities of data.
The results of the efficiency, resolution, charge response, and I-V curves of some of these prototypes were presented at the TREDI2020 Workshop on Advanced Silicon Radiation Detectors~\cite{TREDI2020}.

Collaborating under the guidance of FNAL R\&D scientist Ron Lipton, we performed a series of TCAD and SPICE simulations to explore the plausibility of exploiting the initial transient induced current, described by the Ramo-Shockley Theorem, to achieve the timing precision required for four-dimensional tracking.
The results of these studies were presented at the ULITIMA fast timing conference at Argonne in October 2018~\cite{ULITIMA2018}, and the conference proceedings were published to Nuclear Instrumentation and Methods in Physics~\cite{LIPTON2019162423}.

The contents of this dissertation is outlined as follows.
A brief history of particle physics and a concise overview of the SM is given in Chapter~\ref{Motivation_and_Theoretical_Overview}.
Descriptions LHC and CMS experiment can be found in Chapter~\ref{The_CMS_Experiment_at_the_LHC}.
A summary of top quark physics at hadronic colliders, in particular considerations relevant to the spin properties of \ttbar production and top quark decay, is provided in Chapter~\ref{Top_Quark_Physics_at_the_LHC}. 
An overview of Monte Carlo (MC) event simulation of $pp$ collisions is given in Chapter~\ref{Monte_Carlo_Event_Simulation}.
The particle-flow (PF) algorithm for reconstruction of tracks, vertices, muons, electrons, jets, and missing transverse energy is described in Chapter~\ref{Particle-Flow_Reconstruction}.
Chapter~\ref{Datasets_Event_Selection_Kinematic_Reconstruction} contains details regarding object and event selection, kinematic reconstruction of top quark pairs, background determination, event yields, and data sets used in the measurement.
The unfolding procedure to obtain parton-level differential cross-sections extrapolated to the full phase space and sources of systematic uncertainties are explained in Chapter~\ref{Measurements_of_Differential_Cross-sections}.
Measurement results and the method for extracting spin density coefficients from the differential cross-sections are presented in Chapter~\ref{Results}.
Appendix~\ref{Trigger_Efficiency_Scale_Factors} contains a complete overview of trigger efficiency and corrective scale factor measurements that were used by virtually all CMS Run II measurements analyzing \ttbar events decaying via the dileptonic mode.
Simulation results investigating whether small pixels integrated with fast electronics show promise for realizing the timing resolution required for four-dimensional tracking are presented in Appendix~\ref{Fast_Timing}.

\section{Natural Units}
In this dissertation, some quantities are expressed, when conventional, using the system of natural units.
When using this system, many expressions in physics are simplified by setting the speed of light ($c$) and the reduced Planck's constant ($\hbar$) to unity, and dimensional quantities can be expressed in units of energy or inverse energy.
For example, energy, momentum, and mass can be expressed in units of energy, while length and time can be expressed in units of inverse energy.
In particle physics, energy is typically expressed in units of electronvolts ($\si{\eV}$), which is defined as the amount of kinetic energy gained by a single electron accelerated from rest through an electric potential difference of $\SI{1}{\V}$ in a vacuum.
When these units are used, the standard dimensions can be recovered by appropriately multiplying or dividing by $c$ and $\hbar$.