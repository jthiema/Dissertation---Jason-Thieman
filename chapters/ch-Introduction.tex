\ProvidesFile{chapters/ch-Introduction.tex}

\chapter*{INTRODUCTION}
\addcontentsline{toc}{chapter}{INTRODUCTION}
\label{Introduction}
Experiments to discover and determine the properties of Standard Model (SM) elementary particles are performed at particle accelerator facilities, such as the Large Hadron Collider (LHC) at the European Center for Nuclear Research (CERN), which is the largest and most powerful particle accelerator ever built.
The LHC is capable of producing tens of millions of top quarks from high-energy proton-proton ($pp$) collisions every year, making it a ``top quark factory.''
Due to its unique properties and role in exotic processes, the top quark is important for precision measurements of the SM and is sensitive to new physics (NP) beyond the SM (BSM).
With a very short lifetime caused by its exceptionally large mass, the top quark decays before hadronizing, and its properties, including spin information, are transferred undiluted to its decay products and observable in their kinematic distributions.

This dissertation presents differential measurements of top quark ($t$ and $\bar{t}$) polarizations and top quark pair (\ttbar) spin correlations, which probe the independent coefficients of the top-spin components of the \ttbar production density matrix, using \lumivalueRuniiUL $\pm$ \lumierrRuniiUL\ of data recorded by the CMS experiment at the LHC with \beamenergy, during 2016, 2017, and 2018.
The measurements target all channels (\ee, \emu, \mumu) of the \ttbar dileptonic decay mode with final states containing two oppositely charged leptons.
The measurements follow the analysis strategy of CMS TOP-18-006~\cite{Sirunyan:2681777}, published in Physics Review D, for which I collaborated as a co-author.
Compared to CMS TOP-18-006, the measurement has been extended with additional observables, differential measurements as a function of \ttbar invariant mass, several optimizations that reduce background contributions and systematic uncertainties, and significantly increased luminosity.
With these optimizations and the inclusion of \ttbar events the via $\tau$ decays as signal, signal purity has been increased from \sim$79\%$ to \sim$93\%$ and measurement precision for one-dimensional normalized unfolded cross-sections and extracted coefficients were improved by as much as a factor.
All measured observables are corrected for detector efficiencies, acceptances, and migrations, unfolded to parton-level, and extrapolated to the full phase space using a regularized unfolding procedure with detector response obtained from MC simulated SM predictions with next-to-leading-order (NLO) matrix element accuracy interfaced with parton-shower and hadronization algorithms.
The spin density coefficients are extracted from the unfolded distributions and compared to theoretical predictions, constituting a precision test of the SM.
The measurement results can be used for a multitude of subsequent analyses:
\begin{itemize}
    \item Effective field theory (EFT) interpretation to set limits on dimension-six operators relevant for hadronic \ttbar production~\cite{Sirunyan:2681777}
    \item Supersymmetry (SUSY) interpretation to set limits light supersymmetric top squark production~\cite{CMS-PAS-FTR-18-034}
    \item Observe entanglement in the \ttbar system~\cite{Afik_2021}
    \item Observe a violation of Belle's Inequality in the \ttbar system~\cite{Aguilar_Saavedra_2022}
    \item Search for evidence of Toponium near \ttbar production threshold~\cite{PhysRevD.104.034023}.
\end{itemize}

One of the ingredients for the \ttbar spin correlation measurements is a set of scale factors correcting the trigger efficiencies in MC simulation to those observed in data.
I performed detailed measurements of these scale factors, which were approved by the EGamma and Muon CMS Physics Object Groups and the CMS Top Quark Physics Analysis Group.
The measurements were centrally provided and used by virtually all CMS Run II measurements analyzing \ttbar events decaying via the dileptonic mode.
In order to accommodate a search for Lorentz violation, I collaborated as a coauthor and additionally provided dedicated measurements of the dilepton trigger efficiency scale factors as a function of sidereal time.

My contributions to the operation of the CMS experiment and to R\&D efforts for the upcoming high luminosity LHC (HL-LHC) phase are many-fold.
I have spent many weeks in the CMS Remote Operations Center at Fermi National Accelerator Laboratory (FNAL) under the supervision of distinguished researcher Gabriele Benelli monitoring the data quality of the CMS silicon pixel and strip tracker sub-detectors and certifying collision data sets for tracking quality.
During the long shutdown phase between Runs II and III, I was promoted to on-call expert, and later to shift leader, for the certification of ultra-legacy data sets.
Moreover, I developed data quality monitoring tools for the monitoring of inefficient modules in the silicon strip sub-detector.

Also at FNAL, I contributed to the testing of HL-LHC pixel detector prototypes at the FNAL Test Beam Facility (FTBF) under the supervision of University of Illinois at Chicago physics professor Corrinne Mills.
%The FTBF is an environment to probe the prototypes, consisting of RD-53a readout chips bump-bonded to planar or three-dimensional sensors, under controlled conditions.
%Spills of several hundred thousand protons with energies up to $\SI{100}{\GeV}$ are provided by the accelerator facility every $\SI{60}{\s}$.
%The CMS Tracking Telescope is placed along the beamline and consists of silicon strip and pixel planes constructed from surplus silicon tracker modules of the same design as those currently installed in CMS.
%The prototypes under testing are placed between two sections of planes in the telescope and their performance is probed with a set of beam tracks that are well-identified and reconstructed by the tracking telescope.
Besides data acquisition, I also contributed to the offline analysis of the data by identifying the reconstruction and alignment routines that were most effective and automating these routines in a workflow with built-in quality checks that allowed us to efficiently align the telescope and produce measurement results for large quantities of data.
The results of the efficiency, resolution, charge response, and I-V curves of some of these prototypes were presented at the TREDI2020 Workshop on Advanced Silicon Radiation Detectors~\cite{TREDI2020}.

Collaborating under the guidance of FNAL R\&D scientist Ron Lipton, we performed a series of TCAD and SPICE simulations to explore the plausibility of exploiting the initial transient induced current, described by the Ramo-Shockley Theorem, to achieve the timing precision required for four-dimensional tracking.
The results of these studies were presented at the ULITIMA fast timing conference at Argonne in October 2018~\cite{ULITIMA2018}, and the conference proceedings were published to Nuclear Instrumentation and Methods in Physics~\cite{LIPTON2019162423}.

The contents of this dissertation are outlined as follows:
\begin{itemize}
    \item Chapter~\ref{Motivation_and_Theoretical_Overview} contains a brief history of particle physics and a concise overview of the SM.
    \item Descriptions of the LHC and the CMS experiment are in Chapter~\ref{The_CMS_Experiment_at_the_LHC}.
    \item A summary of top quark physics at hadronic colliders, with particular emphasis on the spin properties of \ttbar production and top quark decay, is set forth in Chapter~\ref{Top_Quark_Physics_at_the_LHC}. 
    \item An overview of $pp$ collision MC event simulation is given in Chapter~\ref{Monte_Carlo_Event_Simulation}.
    \item Chapter~\ref{Datasets_Event_Selection_Kinematic_Reconstruction} contains details regarding object reconstruction and event selection, kinematic reconstruction of top quark pairs, background determination, event yields, and data sets used in the measurement.
    \item The unfolding procedure to obtain parton-level differential cross-sections extrapolated to the full phase space and sources of systematic uncertainties are explained in Chapter~\ref{Measurements_of_Differential_Cross-sections}.
    \item Measurement results and the method for extracting spin density coefficients from the differential cross-sections are presented in Chapter~\ref{Results}.
    \item A summary and outlook are provided in Chapter~\ref{Conclusion}.
    \item Appendix~\ref{Trigger_Efficiency_Scale_Factors} contains a complete overview of trigger efficiency and corrective scale factor measurements that were used by virtually all CMS Run II measurements analyzing \ttbar events decaying via the dileptonic mode.
    \item Simulation results investigating small pixels integrated with fast electronics for realizing the timing resolution required for four-dimensional tracking are presented in Appendix~\ref{Fast_Timing}.
\end{itemize}