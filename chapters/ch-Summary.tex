\ProvidesFile{chapters/ch-Summary.tex}
\begin{refsection}

\chapter{SUMMARY AND OUTLOOK}
\label{Conclusion}

\begin{cabstract}
This final chapter provides a summary of the key findings from the measurement results presented in this dissertation and their implications for the field of particle physics. 
It reflects on the progress made in measuring \ttbar spin correlations as a precision test of the SM. 
The potential for future research by expanding scope of these measurements is considered.
The dissertation concludes with an outlook on the future of \ttbar spin correlation measurements at the anticipated HL-LHC with upgraded detector technology, as well as at proposed lepton colliders.
\end{cabstract}


This dissertation presents the first differential measurements of all the independent coefficients of the top-spin components of the \ttbar production density matrix.
The measurements were conducted using \lumivalueRuniiUL $\pm$ \lumierrRuniiUL\ of data collected by the CMS experiment at the LHC with \beamenergy\ during 2016, 2017, and 2018.
All channels (\ee, \emu, \mumu) of the \ttbar dileptonic decay mode with final states containing two oppositely charged leptons were targeted; 16 observables were probed with a total of 80 differential measurements.
The measured observables were corrected for detector efficiencies, acceptances, and migrations, unfolded to parton-level, and extrapolated to the full phase space using a regularized unfolding procedure with detector response obtained from MC simulated SM predictions with NLO matrix element accuracy interfaced with parton-shower and hadronization algorithms.
The spin density coefficients were extracted from the unfolded distributions and compared to MC simulations and theoretical predictions, constituting a precision test of the SM. 
With significantly increased luminosity, several optimizations that reduce background contributions and systematic uncertainties, and the inclusion of \ttbar events the via $\tau$ decays as signal, signal purity has been increased from \sim$79\%$ to \sim$93\%$ and measurement precision for one-dimensional normalized unfolded cross-sections and extracted coefficients were improved by as much as a factor of two compared to CMS TOP-18-006~\cite{Sirunyan:2681777}.
The measured coefficients showed decent agreement with the MC predictions, but the agreement was improved when compared to QCD perturbative calculations for \ttbar\ production at NLO with electroweak corrections.
The only notable deviations from predictions were observed for the differential measurements of coefficient $B_{1}^{k}$, which may be partially explained by b-quark fragmentation having a significant effect on top quark momentum in the \ttbar ZMF.

This dissertation presents a set of measurements that can be further used for various interpretations. 
For example, a SMEFT interpretation can use these measurements to set limits on dimension-six operators relevant for hadronic \ttbar production~\cite{Sirunyan:2681777}. 
Additionally, the measurements can be used for a SUSY interpretation to set limits on light supersymmetric top squark production or to probe for entanglement~\cite{Afik_2021} or a violation of Belle's Inequality in the \ttbar system~\cite{Aguilar_Saavedra_2022}. 
Furthermore, the measurements can be used to search for evidence of toponium near \ttbar production threshold~\cite{PhysRevD.104.034023}. 
To enable a SMEFT interpretation and fitting of multiple distributions, the top quark research group at Purdue University is determining the statistical and systematic correlation matrices for all bins of all differential cross sections. 
The statistical correlations are estimated using bootstrap resampling of the data, while the systematic correlations are estimated by simultaneously evaluating the systematic variations for all measured bins.

There are various ways to expand measurements of the top-spin components of the \ttbar production density matrix. 
In addition to two-dimensional differential measurements based on the \ttbar invariant mass, differential measurements based on the scattering angle of the top quark in the \ttbar ZMF and the \pT of the top quark are also highly motivated. 
Ideally, three-dimensional differential measurements that consider both the \ttbar invariant mass and top quark scattering angle in the \ttbar ZMF would offer the most effective way to probe perturbative QCD predictions for \ttbar production.
Furthermore, lab frame observables such as $\vert \Delta\phi_{\ell\bar{\ell}} \vert$ and $\vert \Delta\eta_{\ell\bar{\ell}} \vert$ (the difference in azimuthal angle $\phi$ and the difference in pseudorapidity $\eta$ between the two leptons) are indirectly sensitive to the spin density coefficients and can be measured with excellent experimental resolution.

In the future, the \ttbar production density matrix spin coefficients can be measured with the CMS \SI{13.6}{\TeV} data set, and significantly improved precision at the HL-LHC.
The HL-LHC will provide substantially higher luminosity and upgraded detectors that will improve measurement capabilities and reduce systematic uncertainties, including better object identification and resolution, allowing for more precise measurements.

The limited collision energies at lepton colliders have prohibited \ttbar production and measurements of top quark polarizations and \ttbar spin correlations. 
Conducting measurements of the \ttbar production density matrix at the proposed International Linear Collider (ILC) or Compact Linear Collider (CLIC) would probe electroweak \ttbar production, which, unlike QCD production, predicts significant top quark and anti-quark polarizations. 
Thus, measurements at lepton colliders would be complementary to those performed at hadron colliders.

\clearpage
\printbibliography[heading=subbibliography,resetnumbers=true]
\end{refsection}