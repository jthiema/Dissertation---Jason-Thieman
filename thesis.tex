\ProvidesFile{thesis.tex}[2023-02-03 PurdueThesis thesis.tex file]

%
%  The home page for the PurdueThesis software is
%      https://engineering.purdue.edu/~mark/PurdueThesis/
%
%  Be sure to sign up for the PurdueThesis mailing list at
%      https://engineering.purdue.edu/ECN/mailman/listinfo/purduethesis-list
%  so you learn of new versions of this software.  You must be on that
%  mailing list to receive help with this software.
%
%  This is the template root file for an example thesis (for master's
%  degree) or dissertation (for a Ph.D.).  From now on "thesis" will
%  refer to both of these unless stated otherwise.
%
%  LaTeX systems include auxiliary programs to do bibliographies,
%  indexes, etc.  The latexmk program runs the fewest programs needed
%  to update your thesis.  latexmk runs automatically on Overleaf.  If
%  you're LaTeXing your document on a non-Overleaf system you may need
%  to run latexmk manually.
%
%  This thesis contains Feynman diagrams in the ap-physics.tex file.
%  For these to be processed correctly you must use the lualatex
%  program:
%      latexmk -lualatex thesis
%  (If your thesis doesn't have Feynman diagrams---the
%      \ProvidesFile{ap-physics.tex}[2022-10-05 Physics appendix]

\begin{VerbatimOut}{z.out}
\chapter{PHYSICS}
\ix{physics//Physics appendix}

Feynman diagrams\ix{Feynman diagram}
show what happens
when elementary particles collide
\cite{feynman-diagram}.
The Feynman diagrams below are from the
\citetitle{ellis2016} documentation \cite{ellis2016}.
\textbf{%
  You must use \texttt{lualatex} instead
  of \texttt{pdflatex}
  to process documents that use the \texttt{tikz-feynman} package.%
}

The input
in the documentation
is different than here because a different random number generator
is used \cite{menke2019}.
I expect this to be corrected.
In the meantime try replacing \texttt{vertical}
with \texttt{vertical'}
and/or switch some \texttt{fermion}
to \texttt{anti} \texttt{fermion} lines \cite{ellis2017}.
\end{VerbatimOut}

\MyIO


\begin{VerbatimOut}{z.out}
\feynmandiagram [large, vertical'=e to f] {
  a -- [fermion] b -- [photon, momentum=\(k\)] c -- [fermion] d,
  b -- [fermion, momentum'=\(p_{1}\)] e -- [fermion, momentum'=\(p_{2}\)] c,
  e -- [gluon] f,
  h -- [anti fermion] f -- [anti fermion] i,
};
\end{VerbatimOut}

\MyIO


\begin{VerbatimOut}{z.out}
\feynmandiagram [horizontal=a to b] {
  i1 -- [anti fermion] a -- [anti fermion] i2,
  a -- [photon] b,
  f1 -- [fermion] b -- [fermion] f2,
};
\end{VerbatimOut}

\MyIO

%  command may be commented out by prefixing it with a
%  '%') use pdflatex instead of lualatex:
%      latexmk thesis
%
%  To make a final PDF file before you turn in your thesis do
%      latexmk -g thesis
%  This makes sure than everything is done for your final version.
%
%  References cited below:
%
%  TM2017 is short for Thesis Manual 2017:
%    A Manual for the Preparation of Graduate Theses,
%    eighth revised edition,
%    Thesis and Dissertation Office,
%    Purdue University,
%    2017,
%    revised August 30, 2017,
%    http://www.purdue.edu/gradschool/documents/thesis/graduate-thesis-manual.pdf,
%    last retrieved on May, 8, 2021.
%
%  In this file, change the example information to your information.
%

% institution
% Choose an institution name from the following list:
%     VALUE                   COMMENT
%     Purdue University
\def\ZZinstitution{Purdue University}


% campus
% Choose a campus from the following list:
%     VALUE               COMMENT
%     West\space Lafayette
\def\ZZcampus{West\space Lafayette}


% program
% Choose a program from the following list:
%     VALUE
%     Physics
%     Physics and Astronomy
\def\ZZprogram{Physics and Astronomy}

% degree
% Choose a degree from the following list:
%     Doctor of Philosophy
\def\ZZdegree{Doctor of Philosophy}

% author
% Put your name here.
\def\ZZauthor{Jason R. Thieman}

% document
% Choose a document from the following list:
%     A Dissertation
%     A Master's Bypass Report
%     A Preliminary Report
%     A Thesis
\def\ZZdocument{A Dissertation}

% graduation
% Chose a month from
%     May
%     August
%     December
% followed by a space
% then choose a year from 2020 to 2030.
\def\ZZgraduation{May 2023}

% title
% If you need to manually split the title,
% over several lines do, for example,
%     \def\ZZtitle{%
%       This is the First Line\\[-6pt]
%       and this is the Second Line%
%     }
\def\ZZtitle{Differential Measurements of Top Quark Polarization and Spin Correlations in \ttbar Dilepton final states from \pp collisions at \beamenergy with the CMS Experiment}


% showcolophon
% Print the ap-colophon.tex file at the end of the document?
% THE SUBMITTED COPY OF YOUR THESIS MUST BE RUN WITH ZZshowcolophon = {false}.
\def\ZZshowcolophon{false}

% showdiagonalline
% Show a diagonal line from lower left to center
% of main printed part of page?
% THE SUBMITTED COPY OF YOUR THESIS MUST BE RUN WITH ZZshowdiagonalline = {false}.
\def\ZZshowdiagonalline{false}

% showgridlines
% Show grid lines on main printed part of page
% Vertical and horizontal grid lines are put
% in the normal printed part of the page---this
% includes lines where the margins are.
% THE SUBMITTED COPY OF YOUR THESIS MUST BE RUN WITH ZZshowgridlines = {false}.
\def\ZZshowgridlines{false}

% showmarginlines
% Show margin lines on the edge of the normal printed part of the page?
% Margin lines show where the margins are.
% THE SUBMITTED COPY OF YOUR THESIS MUST BE RUN WITH ZZshowmarginlines = {false}.
%     VALUE    MEANING
%     false    don't show marginlines
%     true     show marginlines
\def\ZZshowmarginlines{false}

% showtimestamp
% Show, for example, a "compiled on  2021-03-02  Tuesday  17:16:24"
% timestamp in the upper right corner of page?
%     VALUE    MEANING
%     false    don't show timestamp
%     true     show timestamp
% THE SUBMITTED COPY OF YOUR THESIS MUST BE RUN WITH ZZshowtimestamp = {false}.
\def\ZZshowtimestamp{false}

% todonotes
% Set things up for todonotes.
%     VALUE    MEANING
%     false    don't put todo notes in PDF file
%     true     put 0.8 inch wide todo notes in PDF file
%     wide     put 3.8 inch wide todo notes in PDF file, do not send
%              todonotes = wide output to a printer
% THE SUBMITTED COPY OF YOUR THESIS MUST BE RUN WITH todonotes = {false}.
\def\ZZtodonotes{false}

% Mark Senn uses an `optional-debugging-code.tex file' but does not
% distribute it.  The following line won't do anything if you don't
% have an optional-debugging-code.tex file so you can leave it the
% way it is.
\InputIfFileExists{optional-debugging-code.tex}{}{}

% The \includeonly command can be used to only include some
% files that have \include commands below.  This is handy
% to only include some files so your document will LaTeX
% faster or if you're trying to narrow down where an error
% occurs.  You can use
%     \includeonly{ch-introduction}
% to only include ch-introduction.tex, or
%     \includeonly{ch-introduction,ap-about-appendices}
% to include ch-introduction.tex and ap-about-appendices.tex.
% More files can be added---just put ',' between the names.
% Comment out the following line before submitting the
% final copy of your thesis.
% \includeonly{ch-introduction,ap-about-appendices}

\documentclass{PurdueThesis}


%%%% \ExplSyntaxOn                         %%%% changed 2021-07-27 by mark
%%%% \bool_set_true:N \ZZCenterCaptionB    %%%% changed 2021-07-27 by mark
%%%% \ExplSyntaxOff                        %%%% changed 2021-07-27 by mark

\def\ZZatinformation{}
% If you are at the Hammond or Westville campus
% remove the "%" from the begining of the next line.
%\def\ZZatinformation{~at~Purdue~Northwest}

% If the title contains commas, do, for example,
% \def\ZZtitle{WIRELESS POWER TRANSFER:
% EFFICIENCY, FAR FIELD, DIRECTIVITY, AND PHASED ARRAY ANTENNAS}



% PurdueThesis.cls loads the rotating package which loads the graphicx
% package.  From page 12 of "Packages in the `graphics' bundle", 2021-03-05,
% retrieved 2021-06-16, at https://texdoc.org/serve/grfguide.pdf/0
%     \graphicspath{<dir-list>}
%
%         This optional declaration may be used to specify a list of
%         directories in which to search for graphics files.  The
%         format is the same as for the LaTeX 2e primitive \input@path.
%         A list of directories, each in a {} group (even if there is
%         only one in the list).  For example:
%             \graphicspath{{eps/}{tiff/}}
%         would cause the system to look in the subdirectories eps and
%         tiff of the current directory.  (All modern TeX systems use /
%         as the directory separator, even on Windows.)
%
%         The default setting of this path is \input@path that is:
%         graphics files will be found whereever TeX files are found.
%
% Look in the "graphics" subfolder for graphics files.
% This is done to reduce the number of files in the main thesis folder
% so the ones in there are easier to find.
\graphicspath{{graphics/}}

% Look in the "packages" subfolder for packages.
% This is done to reduce the number of files in the main thesis folder
% so the ones in there are easier to find.
\makeatletter
  \def\input@path{{packages/}}
\makeatother

%
% Configure bibliography.
%
% Automatically configure the bibliography.  Based on the
% institution, campus, and program listed in the \documentclass
% command \ZZBibProcessor is set to "BibLaTeX" or "BibTeX".
% For BibLaTeX, a
%    \usepackage[...]{biblatex}
% is done.  Put your bibliography entries in all-biblatex.bib.
% For BibTeX, a
%     \bibliographystyle{...}
% command is done.  Put your bibliography entries in all-bibtex.bib.
%
% All combinations of institution, campus, and program use BibLaTeX.
% Exceptions that use BibTeX:
%     o  "Purdue University", "West Lafayette", "Earth, Atmospheric,
%        and Planetary Sciences" uses the ametsoc2014 bibliography style.
%     o  "Purdue University", "West Lafayette", "Veterinary Clinical
%        Sciences" uses the ama bibliography style.
%
% To override the default choices picked by \ConfigureBibliography, change,
% for example,
%     \ConfigureBibliography
% to
%     % \ConfigureBibliography
%     \newcommand{\ZZBibProcessor}{BibLaTeX}
%     \usepackage[backend=biber, citestyle=apa, dashed=false, sortcites=true, style=apa]{biblatex}
%     \addbibresource{all-biblatex.bib}
\ConfigureBibliography

%
% This is only done if you are using BibLaTeX.
%
%
% If you don't want to ignore urldate fields,
% comment out (put "%" before) the next ten lines.
%
\DeclareSourcemap
  {
    \maps[datatype=bibtex]
    {
      % Ignore "urldate = {...}" in .bib files.
      % See the first complete example on page 201 of
      %     https://mirrors.rit.edu/CTAN/macros/latex/contrib/biblatex/doc/biblatex.pdf
      \map
        {
          \step[fieldset=urldate, null]
        }
        % Enter approximate (circa) dates using, for example,
        % "year = c2020"  See
        %     https://tex.stackexchange.com/questions/224617/what-is-the-correct-way-to-handle-approximate-dates-in-biblatex
      \map[overwrite=false]
        {
          \step[fieldsource=year]
          \step[fieldset=sortyear, origfieldval, final]
          \step[fieldsource=sortyear, match={c}, replace={}]
        }
    }
  }

% To let {\bfseries\scshape text} work as expected.
% See
%     https://tex.stackexchange.com/questions/27411/small-caps-and-bold-face
\usepackage{bold-extra}

% For chemical figures.
\usepackage{chemfig}

% For typesetting cryptography pseudocode, algorithms, and protocols.
% See
%     https://mirror.las.iastate.edu/tex-archive/macros/latex/contrib/cryptocode/cryptocode.pdf
\usepackage
[
  n,            % or lambda
  advantage,
  operators,
  sets,
  adversary,
  landau,
  probability,
  notions,
  logic,
  ff,
  mm,
  primitives,
  events,
  complexity,
  oracles,
  asymptotics,
  keys,
]
{cryptocode}

% Define
%    \VerbatimInput[options]{filename}
%    \begin{VerbatimOut}{filename} ... \end{VerbatimOut}.
\usepackage{fancyvrb}
  \DefineShortVerb{\|}  % so "|verbatim|" will be verbatim

% For \InpuutIfFileExists.
\usepackage{filehook}

% So "_" will work in URLs when using BibTeX.
\usepackage[T1]{fontenc}

% For nlui testing.
\usepackage{listings}

% For chemical equations.
% See
%     https://ctan.org/pkg/mhchem?lang=en
% From the "Package documentation" linked-to document
%     mhchem needs a couple of other packages.
%     For instance, expl3, amsmath and calc.
\usepackage[version=4]{mhchem}
  % If I'm loading the package to just define a few new commands I'll indent
  % two spaces right after loading the package and define the few new
  % commands here.  If I'm defining more than a few commands I usually do it
  % after loading all the packages.
  % Define "\nitrate" to be the chemical symbol for nitrate.
  \newcommand{\nitrate}{\ce{NO3{-}}}
  % Define "\pnitrate" (short for "parenthesized nitrate") to be the chemical
  % symbol for nitrate surrounded by parentheses.
  \newcommand{\pnitrate}{(\nitrate)}
  % "Define \vpnitrate" (short for "verbose parenthesized nitrate") to be
  % the word "nitrate" followed by a space followed by the chemical symbol
  % for nitrate with parentheses around it.
  \newcommand{\vpnitrate}{nitrate (\nitrate)}

% For
%     \cancel
%     \highlight
% See
%     http://ftp.math.purdue.edu/mirrors/ctan.org/macros/latex/contrib/siunitx/siunitx.pdf
% pages 11--12.
\usepackage{cancel}


% Redefine description, enumerate, and itemize lists.
% See
%     https://mirrors.concertpass.com/tex-archive/macros/latex/contrib/enumitem/enumitem.pdf
% \usepackage{enumitem}
% \setlist[itemize]{leftmargin=7pt,rightmargin=24pt}



% This gets rid of
%     [5] (./thesis.toc
%     ! Undefined control sequence.
%     \vbox_set:Nn ...box:D {\color_group_begin: #2\par
%                                                       \color_group_end: }
%     l.32 ...}Basic Circuit Components}{31}{section.67}
%                                                       %
%     ?
% and
%     [6]
%     ! Undefined control sequence.
%     \vbox_set:Nn ...box:D {\color_group_begin: #2\par
%                                                       \color_group_end: }
%     l.61 ...rline {P.1}Frenchspacing}{67}{section.445}
%                                                       %
%     ?
% errors.
% See
%     https://github.com/latex3/latex2e/issues/73
\usepackage{etoc}

% Define \setmaxprintline{number_of_columns}.
% \usepackage{hardwrap}

% For indexing.  Making an index is optional.
% Make these commands available:
%     COMMAND           DESCRIPTION
%     \index{string}    put "string" in index information
%     \makeindex        save information to make the index
%     \printindex       print the index
% See
%     https://ctan.org/pkg/makeidx?lang=en
% for more information.
\usepackage{makeidx}
  % By default \index ignores its argument.
  % This activates indexing.
  \makeindex
  % The "chapter name" for the index.
  \renewcommand{\indexname}{INDEX}

% The mathtools package
% (see http://mirror.utexas.edu/ctan/macros/latex/required/amsmath/amsmath.pdf)
% loads the amsmath package which defines the
%     align
%     align*
%     alignat
%     alignat*
%     equation
%     equation*
%     flalign
%     flalign*
%     gather
%     gather*
%     multitaper
%     multitaper*
%     split
% environments and extends amsmath by defining many other commands.
% See
%     https://ctan.org/pkg/amsmath
% for information about amsmath and
%     http://ctan.math.washington.edu/tex-archive/macros/latex/contrib/mathtools/mathtools.pdf
% for information about mathtools.
\usepackage{mathtools}

% Define \includemedia.
\usepackage{media9}

% Define \begin{multicols}{number_of_columns} ... \end{multicolumns}.
% Used in ap-text.tex.
\usepackage{multicol}

% Define \ditto.
\usepackage{pa-ditto}

% Define \FigureDash.
% \FigureDash is a dash the width of a digit in the current font.
\usepackage{pa-figure-dash}

% For PurdueThesis, PuTh, TeX, LaTeX, METAFONT, METAPOST, etc. related logos.
\usepackage{pa-logos}

% (Or maybe use isomath instead?  -mark  2021-06-20)
% Follow ISO 80000-2:2019
%     o   put e, i, j, and pi in upright font automatically
%     o   use, for example, "\di x" to get "\,mathrm{d}\/x"
% This loads
%     o   amsmath.sty (which is already loaded above)
%     o   mathtools.sty
%     o   upgreek.sty
% Load the package.
\usepackage{pa-mismath}
  % Tell mismath to put e, i, j, and pi in upright font automatically.
  \enumber
  \inumber
  \jnumber
  \pinumber
  % To typeset math italic e, i, j, and pi use
  %     \mathit e
  %     \mathit i
  %     \mathit j
  %     \itpi

% Define \MyRepeat{what}{repeat}.
% Do "what" "repeat" number of times.
\usepackage{pa-repeat}

% Define \FloatBarrier.
% \FloatBarrier process all unproccesed floats (tables, figures, etc.).
\usepackage{placeins}

% Define \hl.
% Undefine \st so soul will load without an error.
% I hope \st wasn't used for something important!
\let\st\relax
\usepackage{soul}

% Define \textcent.
\usepackage{textcomp}

% !!! This doesn't work yet, figure it out later.
% For \textprimstress.
% \usepackage{tipa}

% Needed for chapter "Graphics", section "TikZ and PGF".
\usepackage{tikz}
  % Needed to customize arrows.
  \usetikzlibrary{arrows.meta}
  % For electrical diagrams.
  % Uses the TikZ package.
  % The circuitikz name is short for "circuit TikZ".
  \usepackage{circuitikz}
  %
  \usepackage{menukeys}
  %
  % Needed for 3D TikZ stuff.
  \usetikzlibrary{3d}
  %
  % Needed for pa-typographic-conventions package.
  \usetikzlibrary{calc,shadows,shapes.misc,shapes.symbols}
  %
  % Needed for putting text along a path.
  \usetikzlibrary{decorations.text}
  %
  % Draw TikZ decorations.
  % Needed for at least the Kalman filter system model graphic.
  \usetikzlibrary{decorations.pathmorphing} % noisy shapes
  %
  % Fit shapes to coordinates.
  % Needed for at least the Kalman filter system model graphic.
  \usetikzlibrary{fit}
  %
  % Draw the background after the foreground.
  \usetikzlibrary{backgrounds}	% drawing the background after the foreground

% Needed for the Feynman diagram in ap-physics.tex.
% Tikz-feynman requires LuaLaTeX instead of pdflatex be run.
% LuaLaTeX screws up spacing in the list of figures so this
% is not loaded and LuaLaTeX should not be used.
\usepackage[compat=1.1.0]{tikz-feynman}

% The vertical space between a table heading and the table contents
% in a tabular environment.
\newcommand{\tabularspace}{\noalign{\vspace*{2pt}}}

% For \sfrac, used to do slanted fractions, similar to, e.g., 1/2,
% but 1 is small and high and 2 is small and low.
\usepackage{xfrac}


\newcommand{\ttbar}{\ensuremath{t\bar{t}}\xspace}
\renewcommand{\pp}{\ensuremath{pp}\xspace}
\newcommand{\beamenergy}{\ensuremath{\sqrt{s}=13~\text{Te\hspace{-.08em}V}}\xspace}
\newcommand{\invfb}{fb$^{-1}$}
\newcommand{\lumivalueRuniiUL}{137.7\xspace \invfb}
\newcommand{\lumierrRuniiUL}{1.6\%}
\newcommand{\GeV}{\ensuremath{\,\text{Ge\hspace{-.08em}V}}\xspace}
\newcommand{\TeV}{\ensuremath{\,\text{Te\hspace{-.08em}V}}\xspace}

\renewcommand{\ee}{\ensuremath{ee}\xspace}
\newcommand{\emu}{\ensuremath{e\mu}\xspace}
\newcommand{\mumu}{\ensuremath{\mu\mu}\xspace}

\newcommand{\metxy}{\ensuremath{E\!\!\!\!/_\text{x,y}}}
\newcommand{\pTmiss}{\ensuremath{\pT^\text{miss}}\xspace}
\newcommand{\ETmiss}{\ensuremath{E_{\mathrm{T}}^{\text{miss}}}\xspace}
\newcommand{\MET}{\ETmiss}
\newcommand{\pT}{\ensuremath{p_{\mathrm{T}}}\xspace}
\newcommand{\HT}{\ensuremath{H_{\mathrm{T}}}\xspace}


% Define \I.
% \I1 does \indent once, \I2 does \indent twice, etc.
\newcommand{\I}[1]{\MyRepeat{\indent}{#1}}

% Define \MyI.
% Typeset my input.
\long\def\MyI#1%
  {%
    {%
      \fontsize{8}{10}\tt
      \VerbatimInput
        [
          firstnumber = 1,
          numbers     = left,
          xleftmargin = 0.33in,
        ]%
        {#1}
    }%
  }

% Define \MyIO.
% Typeset my input and output.
% The input will all be on the same page.
% The output may be split over multiple pages.
\newcommand{\MyIO}
  {%
    \input{z.out}

    {%
      \fontsize{8}{10}\tt
      \VerbatimInput
        [
          firstnumber = 1,
          numbers     = left,
          xleftmargin = 0.33in,
        ]
        {z.out}
    }
    \FloatBarrier
  }

% Define \NL (newline) so LaTeX goes to the next output line.
% Just doing \\ complains
%     ! LaTeX Error: There's no line here to end.
% \mbox{} is an empty math box.
\newcommand{\NL}{\mbox{}\\}

% Print a list of files used and their version numbers in the log file.
\listfiles


% \def\bibindent{0em}
% Customize the bibliography.
% \DefineBibliographyStrings{english}{
%   urlfrom = {URLFROM},
%   urlseen = {URLSEEN}
% }

% For typographical conventions stuff including
%     \Emph{...}
%     \First{...}
%     \Keys{...}
%     \Literal{...}
%     \Menu{...}
%     \Place{...}
%     \Shell{...}
% This must be after
%     \usepackage{tikz}
\usepackage{pa-typographic-conventions}


% For the \begin{example} ... \end{example} environment
% used in ap-linguistics.tex.
\usepackage{covington}
\usepackage{slgloss}

% "CTAN---Comprehensive" did not get hyphenated and extended
% into the right margin when using BibLaTeX and the apa style.
% These did not change it:
%     \hyphenation{Com-pre-hen-sive}
%     \hyphenation{CTAN---Com-pre-hen-sive}
% I changed    publisher = {CTAN---Comprehensive TeX Archive Network},
% to           publisher = {CTAN---Com\-pre\-hen\-sive TeX Archive Network},
% in my all-biblatex.bib file and it worked as expeceted.
% If you need to change the hyphenation points of a word in the text
% you can do, for example,
%     \hyphenation{ve-ry-od-dly-hy-phen-at-ed}


\begin{document}

\setcounter{tocdepth}{3}

\maketitle

% Define front matter
%     dedication
%     acknowledgments
%     preface
%     table of contents
%     list of tables
%     list of figures
%     list of symbols
%     abbreviations
%     nomenclature
%     glossary
%     abstract

%
% Put chapter \include commands here.
%
\ProvidesFile{chapters/ch-Introduction.tex}
\begin{refsection}

\chapter*{INTRODUCTION}
\addcontentsline{toc}{chapter}{INTRODUCTION}
\label{Introduction}

Experiments to discover and determine the properties of Standard Model (SM) elementary particles are performed at particle accelerator facilities, such as the Large Hadron Collider (LHC) at the European Center for Nuclear Research (CERN), which is the largest and most powerful particle accelerator ever built.
The LHC is capable of producing tens of millions of top quarks from high-energy proton-proton ($pp$) collisions every year, making it a ``top quark factory.''
Due to its unique properties and role in exotic processes, the top quark is important for precision measurements of the SM and is sensitive to new physics (NP) beyond the SM (BSM).
With a very short lifetime caused by its exceptionally large mass, the top quark decays before hadronizing, and its properties, including spin information, are transferred undiluted to its decay products and observable in their kinematic distributions.

This dissertation presents differential measurements of top quark ($t$ and $\bar{t}$) polarizations and top quark pair (\ttbar) spin correlations, which probe the independent coefficients of the top-spin components of the \ttbar production density matrix, using \lumivalueRuniiUL $\pm$ \lumierrRuniiUL\ of data recorded by the CMS experiment at the LHC with \beamenergy, during 2016, 2017, and 2018.
The measurements target all channels (\ee, \emu, \mumu) of the \ttbar dileptonic decay mode with final states containing two oppositely charged leptons.
The measurements follow the analysis strategy of CMS TOP-18-006~\cite{Sirunyan:2681777}, published in Physics Review D, for which I collaborated as a co-author.
Compared to CMS TOP-18-006, the measurements have been extended with additional observables, differential measurements as a function of \ttbar invariant mass, several optimizations that reduce background contributions and systematic uncertainties, and significantly increased luminosity.
With these optimizations and the inclusion of \ttbar events the via $\tau$ decays as signal, signal purity has been increased from \sim$79\%$ to \sim$93\%$ and measurement precision for one-dimensional normalized unfolded cross-sections and extracted coefficients were improved by as much as a factor.
All measured observables are corrected for detector efficiencies, acceptances, and migrations, unfolded to parton-level, and extrapolated to the full phase space using a regularized unfolding procedure with detector response obtained from MC simulated SM predictions with next-to-leading-order (NLO) matrix element accuracy interfaced with parton-shower and hadronization algorithms.
The spin density coefficients are extracted from the unfolded distributions and compared to theoretical predictions, constituting a precision test of the SM.
The measurement results can be used for a multitude of subsequent analyses:
\begin{itemize}
    \item Effective field theory (EFT) interpretation to set limits on dimension-six operators relevant for hadronic \ttbar production~\cite{Sirunyan:2681777}
    \item Supersymmetry (SUSY) interpretation to set limits light supersymmetric top squark production~\cite{CMS-PAS-FTR-18-034}
    \item Observe entanglement in the \ttbar system~\cite{Afik_2021}
    \item Observe a violation of Belle's Inequality in the \ttbar system~\cite{Aguilar_Saavedra_2022}
    \item Search for evidence of Toponium near \ttbar production threshold~\cite{PhysRevD.104.034023}.
\end{itemize}

One of the ingredients for the \ttbar spin correlation measurements is a set of scale factors correcting the trigger efficiencies in MC simulation to those observed in data.
I performed detailed measurements of these scale factors, which were approved by the EGamma and Muon CMS Physics Object Groups and the CMS Top Quark Physics Analysis Group.
The measurements were centrally provided and used by virtually all CMS Run II measurements analyzing \ttbar events decaying via the dileptonic mode.
In order to accommodate a search for Lorentz violation, I collaborated as a coauthor and additionally provided dedicated measurements of the dilepton trigger efficiency scale factors as a function of sidereal time.

My contributions to the operation of the CMS experiment and to R\&D efforts for the upcoming high luminosity LHC (HL-LHC) phase are many-fold.
I have spent many weeks in the CMS Remote Operations Center at Fermi National Accelerator Laboratory (FNAL) under the supervision of distinguished researcher Gabriele Benelli monitoring the data quality of the CMS silicon pixel and strip tracker sub-detectors and certifying collision data sets for tracking quality.
During the long shutdown phase between Runs II and III, I was promoted to on-call expert, and later to shift leader, for the certification of ultra-legacy data sets.
Moreover, I developed data quality monitoring tools for the monitoring of inefficient modules in the silicon strip sub-detector.

Also at FNAL, I contributed to the testing of HL-LHC pixel detector prototypes at the FNAL Test Beam Facility (FTBF) under the supervision of University of Illinois at Chicago physics professor Corrinne Mills.
%The FTBF is an environment to probe the prototypes, consisting of RD-53a readout chips bump-bonded to planar or three-dimensional sensors, under controlled conditions.
%Spills of several hundred thousand protons with energies up to $\SI{100}{\GeV}$ are provided by the accelerator facility every $\SI{60}{\s}$.
%The CMS Tracking Telescope is placed along the beamline and consists of silicon strip and pixel planes constructed from surplus silicon tracker modules of the same design as those currently installed in CMS.
%The prototypes under testing are placed between two sections of planes in the telescope and their performance is probed with a set of beam tracks that are well-identified and reconstructed by the tracking telescope.
Besides data acquisition, I also contributed to the offline analysis of the data by identifying the reconstruction and alignment routines that were most effective and automating these routines in a workflow with built-in quality checks that allowed us to efficiently align the telescope and produce measurement results for large quantities of data.
The results of the efficiency, resolution, charge response, and I-V curves of some of these prototypes were presented at the TREDI2020 Workshop on Advanced Silicon Radiation Detectors~\cite{TREDI2020}.

Collaborating under the guidance of FNAL R\&D scientist Ron Lipton, we performed a series of TCAD and SPICE simulations to explore the plausibility of exploiting the initial transient induced current, described by the Ramo-Shockley Theorem, to achieve the timing precision required for four-dimensional tracking.
The results of these studies were presented at the ULITIMA fast timing conference at Argonne in October 2018~\cite{ULITIMA2018}, and the conference proceedings were published to Nuclear Instrumentation and Methods in Physics~\cite{LIPTON2019162423}.

The contents of this dissertation are outlined as follows:
\begin{itemize}
    \item Chapter~\ref{Motivation_and_Theoretical_Overview} contains a brief history of particle physics and a concise overview of the SM.
    \item Descriptions of the LHC and the CMS experiment are in Chapter~\ref{The_CMS_Experiment_at_the_LHC}.
    \item A summary of top quark physics at hadronic colliders, with particular emphasis on the spin properties of \ttbar production and top quark decay, is set forth in Chapter~\ref{Top_Quark_Physics_at_the_LHC}. 
    \item An overview of $pp$ collision MC event simulation is given in Chapter~\ref{Monte_Carlo_Event_Simulation}.
    \item Chapter~\ref{Datasets_Event_Selection_Kinematic_Reconstruction} contains details regarding object reconstruction and event selection, kinematic reconstruction of top quark pairs, background determination, event yields, and data sets used in the measurement.
    \item The unfolding procedure to obtain parton-level differential cross-sections extrapolated to the full phase space and sources of systematic uncertainties are explained in Chapter~\ref{Measurements_of_Differential_Cross-sections}.
    \item Measurement results and the method for extracting spin density coefficients from the differential cross-sections are presented in Chapter~\ref{Results}.
    \item A summary and outlook are provided in Chapter~\ref{Conclusion}.
    \item Appendix~\ref{Trigger_Efficiency_Scale_Factors} contains a complete overview of trigger efficiency and corrective scale factor measurements that were used by virtually all CMS Run II measurements analyzing \ttbar events decaying via the dileptonic mode.
    \item Simulation results investigating small pixels integrated with fast electronics for realizing the timing resolution required for four-dimensional tracking are presented in Appendix~\ref{Fast_Timing}.
\end{itemize}


\clearpage
\printbibliography[heading=subbibliography,resetnumbers=true]
\end{refsection}




% Summary and/or conclusions are optional but often used.
% The summary and/or conclusions often are the last
% the last major division(s) of the text.
% Reference: TM2017 page 32.




% Appendices are optional.  Not all theses contain appendices.
% An appendix is used for supplementary illustrative material,
% original data, computer programs, and other material that is not
% necessarily appropriate for inclusion within the text of your
% thesis.
% Reference: TM2017 page 33.
%
% Use ``\appendix'' for one appendix or ``\appendices'' for more than
% one appendix.
\appendices

\ProvidesFile{chapters/ap-TriggerSF.tex}

\chapter{Measurements of Trigger Scale Factors for Dilepton Final States of \ttbar Events}

\section{Overview and Motivation}
Here is presented the measurements of the trigger efficiencies and corrective scale factors that were used by virtually all CMS Run II measurements analyzing \ttbar events that decay via the dileptonic mode with \ee, \emu, \mumu final states.  
The methods used for the measurements and the determination of systematic uncertainties, were approved by the EGamma and Muon CMS Physics Object Groups, because of the inclusion of electron and muon objects respectively, as well as by the CMS Top Quark Physics Analysis Group.

Trigger efficiencies measure the ability of the CMS trigger system to select potentially interesting events for further analysis.
Differences between the observed trigger efficiencies of data recorded by the detector and the expected trigger efficiencies of simulated data can bias measured physics results, and these differences are corrected via trigger efficiency scale factors that are applied to the trigger efficiencies of simulated data, so that they match the observed trigger efficiencies of recorded data.
Trigger efficiency corrective scale factors are calculated as:
\begin{eqnarray}
SF = \frac{\varepsilon^{data}}{\varepsilon^{MC}}
\label{SF}
\end{eqnarray}
where $\varepsilon^{data}$ and $\varepsilon^{MC}$ are trigger efficiencies for data and Monte Carlo simulated data, respectively.

The method used to measure the dilepton trigger efficiencies utilizes a set of cross-triggers, that are weakly correlated with the dilepton triggers, to count the number of events passing the cross-trigger and \ttbar dilepton final state event selection criteria (denominator), and from that set count also the subset that passes the dilepton triggers (numerator).
Missing transverse energy (\MET) triggers are shown to be minimally correlated with the dilepton triggers, and they provide sufficient statistics for low statistical uncertainties, so they are used as the cross-triggers for these measurements. 
The trigger efficiencies, for both data and Monte Carlo simulated data, are thus calculated as: 
\begin{eqnarray}
\varepsilon = \frac{N_{Events ~passing ~dilepton ~selection, ~\MET ~triggers, ~and ~dilepton ~triggers}}{N_{Events ~passing ~dilepton ~selection ~and ~\MET ~triggers}}
\label{efficiency}
\end{eqnarray}
where, in addition to passing the triggers, events have to also pass the event selection criteria for the appropriate dilepton channel. 

The dilepton trigger efficiency corrective scale factors are measured, separately in the \ee, \emu, \mumu channels, as 1-Dimensional and 2-Dimensional functions of leading and sub-leading lepton \pT.
The 2-Dimensional scale factors are the set recommended to be used by by virtually all CMS Run II measurements analyzing \ttbar events that decay via the dileptonic mode with \ee, \emu, \mumu final states.

\section{Data Sets and Triggers}
The measurements were original performed using pre-Ultra Legacy CMS data sets, but were repeated using the Ultra Legacy sets, which are preserved for long term access, are comprehensively documented, designed to be useful for a wide range of studies, and include high-level physics objects just as reconstructed leptons, jets, and missing transverse energy.
The data streams were collected by the CMS experiment in 2016, 2017 and 2018, and correspond to a combined integrated luminosity of \lumivalueRuniiUL.
The lists of triggers for each of the three run eras were provided in collaboration with the CMS Top Quark Physics Analysis Group trigger experts; both single lepton and dilepton triggers are used in order to maximize the trigger efficiency. 

\section{Object and Event Selection}
The dilepton final state of \ttbar events is characterized by the presence of at least two high-\pT isolated leptons with opposite electric charge , large missing transverse energy (\MET), and two jets resulting from the hadronization of bottom quarks.
The identification and reconstruction of the different objects is performed using the particle-flow (PF) event-reconstruction algorithm, which combines information from all the raw sub-detector signals to accurately identify and reconstruct individual particle objects for each event. 
\subsection{Electrons}
Electron candidates are selected from the reconstructed GSF electrons with $p_{T}>$ 15 \GeV and $\abs{\eta_{SuperCluster}}<$ 2.4, removing the barrel-endcap gap 1.4442 $< \abs{\eta_{SuperCluster}}<$ 1.566. 
Selected electrons must to pass the tight identification tracking requirements, including relative isolation, as recommended by the CMS EGamma Physics Object Group.
Corrections to scale the raw energy measurements and smear the electron resolutions, to match the accuracy and precision of reconstructed electrons in MC simulated data to those in recorded data, are also applied.
\subsection{Muons}
Particle-Flow (PF) muons selected for this measurement are required to be reconstructed as Global Muons, relatively isolated, with $p_{T}>$ 15 GeV and $\abs{\eta}<$ 2.4, and to be identified as passing tight identification tracking criteria recommended by the CMS Muon Physics Object Group.
Corrections to scale the raw energy measurements and smear the muon resolutions, to match the accuracy and precision of reconstructed muons in MC simulated data to those in recorded data, are also applied.
\subsection{Jets}
Particle candidates found by the PF algorithm are clustered into jets using the anti-kT algorithm with distance parameter $R = 0.4$ (AK4). 
Selected jets should have $p_{T} >$ 30 GeV, $\abs{\eta} <2.4$, and pass tight jet identification requirements recommended by CMS Jet-MET Physics Object Group.
In order to remove overlap of selected jets with the selected leptons, jet objects within $\Delta R(jet,lepton)<$ 0.4 of selected leptons are not counted.
\subsection{Missing Transverse Energy}
In order to reduce the instrumental noise in the detector, \MET filters are applied as is recommended by CMS Jet-MET Physics Object Group.
An event selection requirement of $\MET > 100$ is included because of the high threshold of \MET triggers.
\subsection{Event Selection Criteria}
Events with exactly two high-\pT isolated PF leptons (electrons or muons) with opposite electric charge, PF leptons (electrons or muons) are selected. 
Both electrons and muons are required to have $\abs\eta < 2.4$, with $p_{T} >$ 25 \GeV ($p_{T} >$ 15 \GeV) for the leading (sub-leading) lepton.  
The invariant mass of the two leptons is required to be greater than 20 \GeV.  
Selected events are required to have at least one jet passing the loose working point of the DeepCSV b-tagging algorithm.

\section{Statistical and Systematic Uncertainties}
For the 2-Dimensional trigger efficiency scale factor measurements, the systematic uncertainties are added in quadrature with the statistical uncertainties for the final total uncertainties in each bin. 
For the 1-Dimensional trigger efficiency scale factor measurements, the sums in quadrature of the systematic uncertainties are plotted as a hashed uncertainty band for each bin, while the statistical uncertainties are shown as an independent error bar on the data points.
\subsection{Statistical Uncertainties}
The asymmetric statistical uncertainty for measured trigger efficiencies is determined using Clopper-Pearson intervals ~\cite{bib:Cousins:2009kz}.
This is in general a conservative method, since Clopper-Pearson intervals have the property that the actual coverage is at least as big as the nominal one for any population portion. 
It is also described in the statistics section of the Particle Data Group book ~\cite{bib:PDG}.
The asymmetric statistical uncertainty for measured scale factors $T=(X / m) /(Y / n)$ is determined using the method recommended by Katz et al. for ratios of efficiencies, in which $\ln (T)$ is approximately normally distributed \cite{bib:10.2307/2531405}. 
The estimated variance is  $\hat{\sigma}^{2}=(1 / x)-(1 / m)+$ $(1 / y)-(1 / n)$ and an approximate two-sided $1 - \alpha$ confidence interval ($1 \sigma: 0.6827 = 1 - 0.3173$) is given by $\left\{t \exp \left(-\xi_{1-\alpha / 2} \cdot \hat{\sigma}\right), t \exp \left(\xi_{1-\alpha / 2} \cdot \hat{\sigma}\right)\right\}$ where $\xi_{1-\alpha / 2}$ is the $1-\frac{1}{2} \alpha$ fractile of the standard normal distribution, and $t$, $x$, and $y$ are the observed values of $T$, $X$, and $Y$ respectively.  
For the extreme cases in which $x = m$ and $y = n$, $x$ and $y$ are replaced with $x = m - 1/2$ and $y = n - 1/2$.
\subsection{Systematic Uncertainties}
The sources of systematic uncertainties are as follows:
\subsubsection{Event Topology}
In order to estimate the effect of event environment on the trigger scale factors, the data and MC samples are divided into independent regions using number of jets and number of vertices, and scale factors are computed separately in each region. 
An envelope of the largest bin-by-bin differences with respective to the nominal, is taken from the two partitions for number of jets and number of vertices independently.
The final event topology systematic uncertainty in each bin is the sum in quadrature of those two independent sources.
The number of jets distributions is relatively stable for different luminosity conditions during Run II, and the number of jets partition is [$< 3$, $\geq\ 3$] for all three eras.
However, the number of primary vertices distributions are more strongly dependent on the luminosity conditions and a different number of vertex partition is chosen for each era such that roughly the same percentage of MC events are below the partition.
For 2016, 2017, and 2018, the number of vertices partition are [$< 20$, $\geq 20$], [$< 30$, $\geq 30$], and [$< 30$, $\geq 30$] respectively.
\subsubsection{Dependence on Era}
The bin-by bin differences between the nominal measurement of the scale factor and a lumi-weighted combination of scale factors measured for the individual runs in each era is taken as another source of systematic uncertainty.
\subsubsection{Correlations between \MET and dilepton triggers}
We estimate the correlation factor $(\alpha)$ between MET and dilepton triggers in \ttbar MC events by finding the fraction of events passing only the MET triggers ($\epsilon_{trigMET}^{MC}$), only the dilepton triggers ($\epsilon_{trigDL}^{MC}$) and passing both of them ($\epsilon_{trigDL,trigMET}^{MC}$): 
\begin{eqnarray}
\alpha=\frac{\epsilon_{trigDL}^{MC} \times \epsilon_{trigMET}^{MC}}{\epsilon_{trigDL,trigMET}^{MC}} \, .
\label{alpha}
\end{eqnarray}
If \MET and dilepton triggers are independent with no correlation, then $\epsilon_{trigDL,trigMET}^{MC} = \epsilon_{trigDL}^{MC} \times \epsilon_{trigMET}^{MC}$ or $\alpha = 1$ ($P\{A\cap B\} = P\{A\} \times P\{B\}$).
The fraction $(\alpha)$ is determined for each channel and data set from equation \ref{alpha} and its difference from 1 is found to be small for all triggers studied. 
Since these values are negligible with respect to the other systematic uncertainties, they are not added in quadrature for the final total uncertainty. 
Values of $\alpha$ as a function of leading and sub-leading lepton pt, eta, and phi, as well as MET, number of jets, and number of vertices are shown in ~~~. 

\section{Search for Lorentz Invariance Violation in \ttbar Production}
Lorentz symmetry, i.e. that the laws of physics are the same in all inertial frames of reference, is a fundamental principle of general relativity and relativistic quantum field theories.
The search for violation of Lorentz symmetry involves looking for deviations from this symmetry, which could indicate the presence of new physics BSM.
One method to test the validity of Lorentz symmetry is to search for a modulation of the \ttbar cross section as a function of sidereal time (the orientation of the experiment relative to the fixed position of stars in the sky).
In order to accommodate a search and provide time dependence, the dilepton trigger efficiency scale factors were measured 1-Dimensionally as a function of sidereal time using the same method outlined above, with the exception of not including the number of vertices partition as a source of event topology systematic uncertainty.
\subsection{Translating UNIX Time to Sidereal Time}
Every CMS data event is recorded with a UNIX timestamp. 
The sidereal day, which is the time it takes for one rotation of the Earth relative to the stars, is about four minutes shorter than the solar day, which is the time it takes for one rotation of the Earth relative to the Sun.
In UNIX time, the rotational period of the earth (sidereal day) lasts approximately 23h 56min 4s, i.e. 86164 UNIX seconds. 
The sidereal second is defined in such a way that the rotational period of the earth is exactly 24h, i.e. 86400 sidereal seconds.
The following formula is used to translate UNIX time to sidereal time:
\begin{equation}
\Omega_{sidereal} t_{sidereal} = \Omega_{UTC} \times (t_{UNIX} - t_0) + \phi_{UNIX} + \phi_{longitude}
\end{equation}
In this equation:
\begin{itemize}
\item $\Omega_{sidereal}$ is the angular velocity of earth's rotation around its axis in sidereal time: $\Omega_{sidereal} = 2\pi / 86400 s^{-1}$ (sidereal). 
\item On the other hand, $\Omega_{UTC}$ is the angular velocity of earth's rotation around its axis in UTC time: $\Omega_{UTC} = 2\pi / 86164 s^{-1}$ (UTC). 
\item The symbol $t_{UNIX}$ represents the event timestamp recorded at CMS, in UNIX time; this is the number of UTC seconds since the UNIX epoch, i.e. the 1st of January 1970. However, handling large number of seconds is sometimes not practical and a $t_0$ origin is subtracted to set the origin at the first second of 2016.
\item The azimuth angle $\phi_{UNIX}$ encodes the phase between the Unix epoch and "J2000", which is the origin of sidereal time count. J2000 is defined as the direction pointed by the crossing of ecliptic and equator plans, at noon the 1st of January 2000, on the side of the earth were the Sun moves from the Southern hemisphere to the Northern hemisphere. 
\item The azimuth $\phi_{longitude}$ is the effective longitude of the beam at CMS relative to the Greenwich meridian, in radians.
\end{itemize}







% \immediate\setlength{\bibhang}{-3in}
% \immediate\setlength{\itemindent}{3in}
% \immediate\setlength{\rightmargin}{3in}

%
% This is only done if you are using BibLaTeX.
%
\makeatletter  % commented out on 2022-01-26
  \defbibenvironment{bibliography}
    {%
      \list
        {%
          \printtext[labelnumberwidth]%
          {%
            \printfield{prefixnumber}%
            \printfield{labelnumber}%
          }%
        }%
        {%
          \setlength{\bibhang}{1in} %%%%% was 0pt
          \setlength{\itemindent}{1in}%  -\leftmargin} %%%%% was 0pt
          \setlength{\itemsep}{\bibitemsep}%
          \setlength{\leftmargin}{0pt}%  .22in} % 0.42in}
          \setlength{\parsep}{\bibparsep}%
           \setlength{\rightmargin}{0.33in}%
        }%
    }
    {\endlist}
    {\item}
\makeatother  % commented out on 2022-01-26

% \immediate\setlength{\labelnumberwidth}{1.5in} %%%%% was commented out
\setlength{\labelwidth}{1.5in}
\def\sllnsez{[999] }

{%
  % Make _ in URLs visible.
  % \def\t{\char'137}%
  \catcode`*=\active
  \def*{\char'137}%  \char'137 is _
  \PrintBibliography
}


% My filename conventions:
%     FILE THAT START WITH    ARE
%     ap-                     appendices
%     ch-                     chapters
%     gr-                     graphics
%     pa-                     packages
%     z                       temporary files

% LaTeX won't read after the \end{document} command.
% You can put notes to yourself or LaTeX input not
% ready for use after "\end{document}" if you'd like.
\end{document}
